\UseRawInputEncoding
\documentclass{article}

% Language setting
% Replace `english' with e.g. `spanish' to change the document language

% \usepackage[deutsch]{babel}
\usepackage[ngerman]{babel}

%\usepackage[ansinew]{inputenc}
\usepackage[utf8]{inputenc}

% nice code input for appendix
\usepackage{tcolorbox}
\tcbuselibrary{minted,breakable,xparse,skins}

% Replace `letterpaper' with `a4paper' for UK/EU standard size
\usepackage[letterpaper,top=2cm,bottom=2cm,left=3cm,right=2cm,marginparwidth=1.75cm]{geometry}
\usepackage[utf8]{inputenc}
% Useful packages
\usepackage{amsmath}
\usepackage{graphicx}
\usepackage[colorlinks=true, allcolors=black]{hyperref}
% for pictures ... reason: https://tex.stackexchange.com/questions/8625/force-figure-placement-in-text
\usepackage{float}

% highlighted code in text
\usepackage{soul}
\definecolor{Light}{gray}{.90}
\sethlcolor{Light}
\let\OldTexttt\texttt
\renewcommand{\texttt}[1]{\OldTexttt{\hl{#1}}}

% einrücken abschalten
\setlength{\parindent}{0pt}

\begin{document}

\begin{titlepage}
		\begin{center}
			\vspace*{1cm}
			
			\Huge
			\textbf{Spark \\Type-Token-Ratio}
			
			\vspace{2cm}
			
			\Large
                Programmierkonzepte und Algorithmen WS25/26\\
				Justin Gebert, Ole Lordieck\\
   
			\vfill
			
			\vspace{0.8cm}
			
			\includegraphics[width=0.3\textwidth]{htw_logo.jpg} 
			
			\vspace{1cm}
			
			Hochschule für Technik und Wirtschaft Berlin\\
			Fachbereich 4\\
			Informatik, Kommunikation und Wirtschaft
			
			
		\end{center}
	\end{titlepage}

\newpage

\tableofcontents
\thispagestyle{empty}  % no page number

\newpage

%------------------------------------------------------------------------------
% START CONTENT
%
%\setcounter{page}{1}  % start counting pages after table of contents

\section{Aufgabenstellung}

Die Aufgabe besteht darin, einen gegebenen Textkorpus mithilfe von Apache Spark verteilt zu verarbeiten, um die sprachliche Vielfalt der enthaltenen Texte zu untersuchen. Dazu sollen die Texte zunächst als normale Textdateien eingelesen. Anschließend soll mit Spark die Type-Token-Ratio (TTR) für jede im Datensatz vorkommende Sprache berechnet werden. Die TTR wird mit der Anzahl der einzigartigen Wörter durch die Gesamtzahl aller Wörter berechnet. 
\[
\text{TTR} = \frac{\text{Anzahl aller Wörter}}{\text{Anzahl einzigartiger Wörter}}
\]
Außerdem sollen vor der Analyse alle Stoppwörter entfernt werden, um aussagekräftigere Ergebnisse zu erhalten. Ziel ist es, die sprachliche Vielfalt zwischen den Sprachen zu vergleichen und gleichzeitig die Vorteile verteilter Big-Data-Verarbeitung sichtbar zu machen, indem der Datensatz bei Bedarf künstlich vergrößert wird.



\section{Lösungsbeschreibung}

\section{Beschreibung des Codes}

\section{Screenshots der Ergebnisse}

\section{Tests}

\section{Leistung und Laufzeiten}

\section{Fazit} 

\newpage



%------------------------------------------------------------------------------
% BIBLIOGRAPHY AND FIGURE LIST
%
%\bibliographystyle{ieeetr}
%\bibliography{sample}
%\listoffigures
%------------------------------------------------------------------------------

\clearpage
\appendix

%------------------------------------------------------------------------------
% EIDESSTATTLICHE ERKLÄRUNG
%
%\newpage
%
%\section*{Eidesstattliche Erklärung}
% Hiermit erkläre ich, dass ich die vorliegende Arbeit selbstständig und eigenhändig sowie ohne unerlaubte fremde Hilfe und ausschließlich %unter Verwendung der aufgeführten Quellen und Hilfsmittel angefertigt habe.\\
% 
% \noindent Die selbständige und eigenständige Anfertigung versichert an Eides statt:
%
% \vspace{3cm}
% % Text über den Linien
% \noindent\hspace{1,05cm} Berlin, 11. Juli 2023\hfill\includegraphics[width=115pt]{signature.jpg} \hspace{1cm}
% \vspace{-0,3cm} % Anpassen der Höhe des Ort, Datum und Unterschrift
% 
% % Linien
% \noindent\hspace{1cm} \rule{3.25cm}{0.5pt} \hfill \rule{4cm}{0.5pt} \hspace{1cm}
% \vspace{-0,1cm} % Anpassen der Höhe des Textes unter den Linien
% 
% % Text unter den Linien
% \noindent\hspace{1,85cm} Ort, Datum \hfill Unterschrift \hspace{2cm}
%
%------------------------------------------------------------------------------

\end{document}
